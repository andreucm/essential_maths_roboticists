\section{Computer Vision Geometry}
\label{sec:computer_vision_geometry}
This section details the geometry behind the camera projections, so how 3D elements (points, lines, planes) transform from different projective views, as those taken from monocular cameras.

\subsection{Homogeneous points}
A point in 2D (or 3D) is usually represented as a vector of 2 (or 3) components, each one representing the coordinate value of the point. However there exists a useful representation consisting on adding an extra component. For a 2D point, its homogeneous representation holds the following equality: 
\begin{equation}
 \mathbf{x} = 
 \left[
 \begin{array}{c}
  x\\
  y\\
  w\\
 \end{array}
\right] = 
\left[
 \begin{array}{c}
  x/w\\
  y/w\\
  1\\
 \end{array}
\right]
\end{equation}
meaning that a 2D homogeneous point can be also interpreted as all 3D points in the line joining the origin and the point $[x/w\ y/w\ 1]$, so the 2D point $[x/w\ y/w]$ at plane $z=1$. 

\subsection{Homogeneous 2D lines}

\subsection{Homogeneous 2D point and lines relations}

\subsection{Homogeneous 3D lines}
TO DO ...


\subsection{Homogeneous 3D planes}
TO DO ...



\subsection{Pinhole camera model}
Pinhole model tries to parametrize the projection made by a monocular camera of the 3D world in a compact and linear way. Monocular cameras equipped with real lenses are far of being pinhole devices, but camera calibration is the process of \textit{pinholing} such real devices. 

TO DO: add picture of the model

The compact parameterization of the pinhole model is the camera matrix $\mathbf{K}$. It  stablishes a linear relation between homogeneous 3D metric points and their projection to image in pixels units. It has the following parameters: 
\begin{equation}
 \mathbf{K} = 
 \left[
 \begin{array}{ccc}
  f_x & 0 & c_x \\
  0 & f_y & c_y \\
  0 & 0 & 1 \\
 \end{array}
\right]
\end{equation}
where $f_x,\ f_y$ are the focal length of the pinhole model multiplied by the meters to pixels ratio of the camera sensor. These two factors are usually slightly different in real lenses/camera pairs. $c_x,\ c_y$ are the image center offsets in pixels to center the projection into the image. They are close to the half of the horizontal and vertical image sizes respectively. 

\subsection{3D point projection}
A given 3D homogeneous point $\mathbf{x}^C$, expressed in the camera reference frame $\mathcal{F}^C$, can be easily projected to the image thanks to the $\mathbf{K}$ matrix as follows: 
\begin{equation}
 \left[
 \begin{array}{c}
  u\\
  v\\
  1\\
 \end{array}
 \right] = 
 \left[
 \begin{array}{cccc}
  f_x & 0 & c_x & 0\\
  0 & f_y & c_y & 0\\
  0 & 0 & 1 & 0\\
 \end{array}
 \right]
 \left[
 \begin{array}{c}
  x\\
  y\\
  z\\
  1\\
 \end{array}
 \right]
\end{equation}
or in a more compact way: 
\begin{equation}
\label{eq:general_projection}
 \mathbf{u} = \left[\mathbf{K} \vert \mathbf{0}_{3x1} \right] \mathbf{x}
\end{equation}
In both equations above, $\mathbf{u}$ is the 2D homogeneous point in the image corresponding to 3D point $\mathbf{x}$, and $u$ and $v$ are its horizontal and vertical coordinates in pixels.

\subsection{2D image point back-projection}
Given an homogeneous version of an image point in pixels $\mathbf{u}=(u,v,1)^T$, there are infinite 3D points that may correspond to it, $\mathbf{q}$. All these infinte 3D points lay on a ray starting at the camera center passing through the image point at image plane and going up to the infinity. In homogeneous coordinates , the weighted sum of two points is a point somewhere (depending on the weight) in the line joining them, so: 
\begin{equation}
\label{eq:back_projection}
\mathbf{q}^W(\lambda) = \mathbf{P}^{+}\mathbf{u} + \lambda \mathbf{c}^W
\end{equation}
where $\mathbf{c}^W$ is the 3D homogeneous point of the camera center with respect to the wolrd frame, $\mathbf{P}^{+}$ is the right pseudo-inverse matrix of $\mathbf{P}$ and $\lambda$ is an aribitrary real number in $[-1,\infty]$, so the line joining $\mathbf{q}^W(\lambda)$ and $\mathbf{c}^W$ becomes a ray syarting at camera center ($\lambda = \infty$) and finishing at the infinity ($\lambda=-1$). The matrix $\mathbf{P}$ is: 
\begin{equation}
\label{eq:matrix_P}
 \mathbf{P} = \left[\mathbf{K} \vert \mathbf{0}_{3x1} \right] \mathbf{T}^C_W
\end{equation}
with $\mathbf{T}^C_W$ is the rigid transformation of world frame with respect to camera frame, and the following fulfills : 
\begin{equation}
\label{eq:matrix_P_Pinv}
\begin{array}{c}
 \mathbf{u} = \mathbf{P} \mathbf{q}^W; \\ 
 \mathbf{P}^+\mathbf{u} = \mathbf{P}^+\mathbf{P} \mathbf{q}^W;\\
 \mathbf{P}^+\mathbf{u} = \mathbf{I} \mathbf{q}^W;\\
 \mathbf{P}^+\mathbf{u} = \mathbf{q}^W;\\
 \end{array}
\end{equation}

Since the back-projection of an image point corresponds to a 3D ray, and not to a single 3D point, the parameter $\lambda$ is not constrained. If we know at least one coordinate of the 3D point (typically $z$ for detpth cameras), then we can solve for $\lambda$ and find the unique 3D point as: 







\subsection{3D line projection}

\subsection{3D plane projection}

\subsection{Homography mapping}
An homography is the linear mapping existing between points lying on two diferent planes. It is represented by a $3\times 3$ matrix $\mathbf{H}$, which allows to pass from a point in one plane to its corresponding point in the other. It is typically used to stablish relations between points lying on a phisycal 3D plane to their correspondences in the image plane, or between two images of the same planar surface. 

The matrix $\mathbf{H}$ has $8$ degrees of freedom (dof), since it is normalized somehow (for instance, $h_{22}=1$ or $\sum_{ij} h^2_{ij} = 1$. From the matrix $\mathbf{H}$ obtained of two images of the same planar surface, the camera pose can be retrieved. 

The following procedure deduces the homography between two images of the same planar surface, from the general projection equation~\ref{eq:general_projection}. 

\subsection{Fundamental Matrix} 

\subsection{Intrinsics Calibration} 

\subsection{Extrinsics Calibration} 

